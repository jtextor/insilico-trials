\documentclass[tikz]{standalone}


\pagestyle{empty}

\usepackage{fontspec}
\IfFontExistsTF{Helvetica Neue Light}{
\setmainfont{Helvetica Neue Light}
}{
}

%\usepackage{amsmath}
\usepackage{tikz}
\usepackage{graphicx}

% Font settings:
%\renewcommand{\familydefault}{\sfdefault}

\newcommand{\figf}{} %Defines the font used for the labelling of figure panels.


\newcommand{\panellabel}[2]{
\node [inner sep=1pt,anchor=north west] at (#2) {\figf #1};
}
\newcommand{\panel}[3]{
\node [inner sep=1pt,anchor=north west,align=center] (p#1) at (#2) {#3};
\panellabel{#1}{#2};
}

\begin{document}

\footnotesize

\begin{tikzpicture}[yscale=1]

\panel{A}{0,0}{\includegraphics[page=1]{plots/power-analysis.pdf}}
\panel{B}{5.667,0}{\includegraphics[page=5]{plots/power-analysis.pdf}}
\panel{C}{11.337,0}{\includegraphics[page=3]{plots/power-analysis.pdf}}
%\panel{C}{0,-4}{\includegraphics[page=3]{plots/power-analysis.pdf}}
%\panel{D}{7,-4}{\includegraphics[page=4]{plots/power-analysis.pdf}}

\begin{scope}[xshift=2cm,yshift=-5.5cm]
\node at (4,0) {Effect size:}; 
\node (lr) [anchor=east] at (7,0) {2-year OS};
\node (cs) [anchor=west] at (8,0) {Hazard ratio};

\draw [line width=1pt] (lr.west) -- +(-5mm,0);
\draw [line width=1pt,red] (cs.west) -- +(-5mm,0);
\end{scope}

\end{tikzpicture}


\begin{tikzpicture}[yscale=1]

\panel{A}{0,0}{\includegraphics[page=1]{plots/power-analysis.pdf}}
\panel{B}{5.667,0}{\includegraphics[page=2]{plots/power-analysis.pdf}}
\panel{C}{11.337,0}{\includegraphics[page=3]{plots/power-analysis.pdf}}

\begin{scope}[yshift=-5.25cm]
\panel{D}{0,0}{\includegraphics[page=4]{plots/power-analysis.pdf}}
\panel{E}{5.667,0}{\includegraphics[page=5]{plots/power-analysis.pdf}}
\panel{F}{11.337,0}{\includegraphics[page=6]{plots/power-analysis.pdf}}
\end{scope}


\begin{scope}[xshift=2cm,yshift=-10.5cm]
\node at (4,0) {Effect size:}; 
\node (lr) [anchor=east] at (7,0) {2-year OS};
\node (cs) [anchor=west] at (8,0) {Hazard ratio};

\draw [line width=1pt] (lr.west) -- +(-5mm,0);
\draw [line width=1pt,red] (cs.west) -- +(-5mm,0);
\end{scope}

\end{tikzpicture}

\end{document}

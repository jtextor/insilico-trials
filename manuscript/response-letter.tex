\documentclass[10pt]{article}

\usepackage{a4wide}

\usepackage{fontspec}
\setmainfont{Times New Roman}

\usepackage{enumerate}

\usepackage{parskip}



\begin{document}

Dear dr. Righetto, \vspace{10pt}

We are offering a revised version of our manuscript ``\emph{In silico} cancer immunotherapy trials uncover the consequences of therapy-specific response patterns for clinical trial design and outcome'' for consideration at \emph{Nature Communications}.

We would like to thank the four reviewers for their detailed and insightful comments on our manuscript.
We have made the following major changes to the manuscript to address their points and concerns: 

\begin{enumerate}[(I)]

	\item Reviewer 1 asked:  ``would different models lead to similar results?'' We have added 
		two additional mathematical models of cancer immunotherapy
		by other authors to the manuscript and we show that despite
		considerable differences between the models, the results are indeed broadly similar.
		We think this was indeed a very important change, since in reality there will always
		be considerable uncertainty as to which model is the most appropriate one, and it may
		therefore be advisable to consider predictions of multiple models.

	\item Several reviewers commented about specific aspects of our model (such as the exponent
		used to represent tumor growth rate slowdown). We changed these aspects as suggested by
		the reviewers and re-did all analyses, which reassuringly did not change our main 
		results and conclusions. 

	\item Several reviewers pointed out that we were incorporating patients from a lung cancer cohort 
		into our analysis as if they were untreated. We now account for a treatment effect in this
		cohort. Again, this did not alter our conclusions.

	\item To make the methodology proposed in this paper more readily useable, we provide both a simple
		web-based implementation (usable by anyone with a web browser) and a fast R package to perform
		\emph{in silico} immunotherapy trials using the three mathematical models we implemented.

\end{enumerate}

These and other changes are described and presented in one new figure, four new supplementary figures, and two new
tables. All other tables and figures (except only Figure 1) have been completely re-made to accommodate the changes
to our model as well as the two new models. 

All further specific points raised have been addressed as described in our point-by-point response.		

In summary, we feel that the reviewers' input has considerably improved our manuscript, and we have considered and addressed every point (for details, see the point-by-point response). We hope that with these revisions, the manuscript is now suitable for publication in Nature Communications. \vspace{10pt}

Kind regards,

Johannes Textor (on behalf of all authors)

\end{document}
